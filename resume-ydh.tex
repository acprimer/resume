%%%%%%%%%%%%%%%%%%%%%%%%%%%%%%%%%%%%%%%%%
% "ModernCV" CV and Cover Letter
% LaTeX Template
% Version 1.11 (19/6/14)
%
% This template has been downloaded from:
% http://www.LaTeXTemplates.com
%
% Original author:
% Xavier Danaux (xdanaux@gmail.com)
%
% License:
% CC BY-NC-SA 3.0 (http://creativecommons.org/licenses/by-nc-sa/3.0/)
%
% Important note:
% This template requires the moderncv.cls and .sty files to be in the same 
% directory as this .tex file. These files provide the resume style and themes 
% used for structuring the document.
%
%%%%%%%%%%%%%%%%%%%%%%%%%%%%%%%%%%%%%%%%%

%----------------------------------------------------------------------------------------
%	PACKAGES AND OTHER DOCUMENT CONFIGURATIONS
%----------------------------------------------------------------------------------------

\documentclass[zh,color,11pt,a4paper,sans]{moderncv} % Font sizes: 10, 11, or 12; paper sizes: a4paper, letterpaper, a5paper, legalpaper, executivepaper or landscape; font families: sans or roman

\moderncvstyle{casual} % CV theme - options include: 'casual' (default), 'classic', 'oldstyle' and 'banking'
\moderncvcolor{blue} % CV color - options include: 'blue' (default), 'orange', 'green', 'red', 'purple', 'grey' and 'black'

\usepackage{lipsum} % Used for inserting dummy 'Lorem ipsum' text into the template

\usepackage[scale=0.75]{geometry} % Reduce document margins
\usepackage{amssymb}
%\setlength{\hintscolumnwidth}{3cm} % Uncomment to change the width of the dates column
%\setlength{\makecvtitlenamewidth}{10cm} % For the 'classic' style, uncomment to adjust the width of the space allocated to your name
\usepackage{fontspec}
\usepackage{xunicode}
\usepackage{xeCJK}
\setmonofont{Courier New}
\setCJKmainfont{SimSun}
\setCJKsansfont{KaiTi}
\setCJKmonofont{SimHei}
%----------------------------------------------------------------------------------------
%	NAME AND CONTACT INFORMATION SECTION
%----------------------------------------------------------------------------------------

\firstname{} % Your first name
\familyname{姚大海} % Your last name

% All information in this block is optional, comment out any lines you don't need
\mobile{(+86) 152-1072-4792}
\email{acprimer.yao@gmail.com}
%\photo[70pt][0.4pt]{pictures/picture} % The first bracket is the picture height, the second is the thickness of the frame around the picture (0pt for no frame)
\quote{(+86) 152-1072-4792 \quad $\bullet$ \quad acprimer.yao@gmail.com\\https://github.com/acprimer}
\homepage{https://github.com/acprimer}
%----------------------------------------------------------------------------------------

\begin{document}

\makecvtitle % Print the CV title

%----------------------------------------------------------------------------------------
%	EDUCATION SECTION
%----------------------------------------------------------------------------------------

\section{求职意向——软件工程师实习生}

\section{教育背景}

\cventry{2013--2016}{北京航空航天大学}{北京}{\quad 计算机硕士学位在读}{\quad 研究生一等奖学金}{}  % Arguments not required can be left empty
\cventry{2009--2013}{华北电力大学}{北京}{\quad 计算机学士学位}{\quad 本科四年专业前10\%}{}

%----------------------------------------------------------------------------------------
%	WORK EXPERIENCE SECTION
%----------------------------------------------------------------------------------------
%------------------------------------------------


	
\section{项目经验}

\subsection{实验室项目}

\cventry{2014.4--至今}{Android开发负责人}{ACT实验室}{北京}{}{参与国家973项目子课题iCrowd(众智北航)项目Android客户端的开发并担任负责人。该项目主要面向北航师生,为他们提供一个发布新鲜事和实时问答的平台。该项目的显著特色就是利用群体的智慧解决大家生活学习过程中碰到的各种难题,比如找自习室、二手书、收包裹等。该项目已经得到校方多个部门认可并在北航校内进行推广试验。项目主页:http://www.service4all.org.cn/iCrowd/
	\newline{}
	具体工作:
	\begin{itemize}
		\item 在北航校园内针对学生进行初步的市场调研和需求分析
		\item 参与项目后台基础开发和搭建
		\item 负责Android平台项目开发,实现应用基础功能
		\item 进行产品的用户体验调研,根据需求进行应用版本迭代
	\end{itemize}}
	

\cventry{2014.3--4}{商品评论系统开发}{ACT实验室}{北京}{}{参与实验室科研项目商品评论系统网站的开发。该项目主要解决京东等电商的商品评论分级混乱的现状,通过集合群体的智慧对评论的等级进行重新划分。项目主页:http://www.service4all.org.cn/crowdsourcing/ratings/
	\newline{}
	具体工作:
	\begin{itemize}
		\item 采用MySQL进行后台数据库的设计与实现
		\item 采用JSP技术开发网站后台
	\end{itemize}}
	

\subsection{实习经验}
\cventry{2013.3--9}{Android开发实习生}{\textsc{北京睿伍行至科技有限公司}}{北京}{}{在实习期间,完整参与了公司核心项目---3GO免费流量的设计开发、产品推广和上线发布等一系列工作,学习了大量的Android开发知识,接触了产品开发的流程,为以后工作打下良好基础。项目主页:http://3go.3gjiayou.com/\newline{}
具体工作:
\begin{itemize}
\item 采用Axure工具参与应用原型设计
\item 独立开发实现应用部分功能模块,包括推荐功能、账户管理等
\item 独立设计与开发实现应用内嵌小游戏——猜诗词
\end{itemize}}
%------------------------------------------------

\section{研究方向}

\subsection{北京航空航天大学计算机学院新技术研究所(ACT)}

\cvitem{项目}{参与国家973项目子课题iCrowd(众智北航)项目Android平台开发}
\cvitem{科研}{研究Android平台移动应用恶意行为检测}


%----------------------------------------------------------------------------------------
%	AWARDS SECTION
%----------------------------------------------------------------------------------------

\section{获奖荣誉}
\cvitem{2014}{OW2国际软件编程比赛一等奖,参赛作品为iCrowd(众智北航)}
\cvitem{2014}{北航研究生一等奖学金,专业前10\%}
\cvitem{2012}{第37届\textsc{ACM/ICPC}国际大学生程序设计竞赛成都站银牌,杭州站铜牌}
\cvitem{2012}{美国大学生数学建模竞赛\textsc{MCM}一等奖}
\cvitem{本科期间}{多次获得国家奖学金和校级一等奖学金}

%----------------------------------------------------------------------------------------
%	COMPUTER SKILLS SECTION
%----------------------------------------------------------------------------------------

\section{英语技能}

\cvitem{基础}{英语六级(542)}
\cvitem{能力}{有较强的英文论文和技术文档阅读能力,有良好的交流沟通能力}

\section{IT技能}

\cvitem{基础}{熟练使用\textsc{Java/C},熟悉常用数据结构和算法}
\cvitem{}{熟悉\textsc{Jsp}和\textsc{MySQL}等技术}
\cvitem{经验}{熟悉\textsc{Android}开发,有一年的开发经验}
\cvitem{}{有\textsc{Jsp}网站开发经验,使用过\textsc{Struts/Hibernate}等框架}

%----------------------------------------------------------------------------------------
%	INTERESTS SECTION
%----------------------------------------------------------------------------------------

\section{自我评价}
\cvitem{}{编程基础扎实,算法基础好,有ACM/ICPC程序竞赛经历。}
\cvitem{}{自主学习能力较好,数理基础好,能够很快接受新的知识。}
\cvitem{}{对技术有热情,喜欢钻研,有团队合作能力,能迅速融入团队。}


%----------------------------------------------------------------------------------------
%	COVER LETTER
%----------------------------------------------------------------------------------------

% To remove the cover letter, comment out this entire block
%----------------------------------------------------------------------------------------

\end{document}